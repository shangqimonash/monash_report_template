\chapter{Research Problem}
This report outlines the doctoral research on solving sensitive data privacy issue in Online Social Network (OSN) by using 'searchable encryption' technique. 

\section{Background of Problem}
OSN is highly popular for many years, and more and more Internet users start to use OSN as their medium of communication: The monthly active user of Facebook is reached 2 billion in 2017, which means over one half Internet users are using the services from Facebook \cite{statista2017facebook, itstat2017population}, and this is just an epitome of the today's OSN dependency.

A result of the huge growth of OSN is more and more sensitive personal data is revealed while OSN users enjoy using it. In 2005, two researchers from Carnegie Mellon University conducted a study in the online behaviour of more than 4,000 students in the university who had signed up in Facebook \cite{gross2005information}. In their study, they found that the majority of Facebook users at CMU provided astonishing amount of sensitive information in their Facebook profiles: it may contain their real names (over $90\%$), date of birth ($87\%$), personal images ($80\%$) and so on; actually, privacy preference is provided to OSN user for controlling the visibility and access privilege of their profiles, but it is sparingly used, which makes most of the users searchable and identifiable to the vast majority of anybody else in the network. 

Due to the richness and variety of sensitive information in OSN, the privacy of these data is of critical importance to OSN users, because the permissive and unconcerned uses of sensitive information will put themselves at risk for many attacks from the cyberspace or even the real world: The precise personal information collecting from OSN helps potential adversaries to construct more deceitful fraud messages (e.g. Context-Aware Spam \cite{brown2008social}); it also can be used to re-identify some anonymous datasets like hospital medical information by matching the common attributes \cite{sweeney2002k}; additionally, OSN users may be easily stalked as they revealed their location or timetable when they shared their activities with their friends.

Although the main purposes of OSN are to help its members to communicate with others and maintain social relations in a more convenient way, OSN service providers also design diverse Social Network Services (SNS) for its user interaction virtually: it includes updating activities and location information, sharing multimedia (e.g. photo, video) during some events, getting updates and comments on activities by friends, etc.. All of these services retain a huge amount of data about its users: Until 2013, there were 1.11 billion monthly active users and they put 4.75 billion shared items in Facebook, these contents received over 4.5 billion tags such as 'Likes' from their friends \cite{facebook2013growth}. 

Making these data searchable to further enhance to capabilities of OSN had become a prevailing trend in recent years. In 2013, Facebook introduced their social data search engine -- Graph Search \cite{facebook2013graph}. This search engine aims to return the information based on the content from user's social network of friends and connections (their social circle): For example, users may search ``The restaurants visited by friends live in Melbourne'' to get restaurants recommendation from their friends in Melbourne. The search results are more personally relevant and satisfactory because the results are generated and refined by users' social circle where the people within always have some similar or the same attributes (e.g. geographical location, hobbies, education background etc.) according to ``Homophily'' theory \cite{mcpherson2001birds}. The concept of Graph Search is also introduced by other OSN service provider: LinkedIn has Job Search Engine which is based on users' location, skill as well as the information from their colleague and alumni; WeChat also can search user-specific content from Moments and Official Accounts. 

In a nutshell, Graph Search-like search engines are powerful tools from the aspect of search because the search results are from a more reliable source (i.e. social circle) and more user-centric. However, they make user data searchable and then provide an extremely simple way to unearth user sensitive data: For instance, potential adversaries now can easily construct a query like ``Friends in Melbourne working/traveling outsides'' to find break-in crime targets. As a result, these new social data search engines (``Graph Search'' is used to represent them) cause increasing privacy concerns about OSN to the public, and, to protect user's privacy in the context of Graph Search becomes the main motivation of this research.

\section{Research Questions}
In general, this research is going to answer following question:
\begin{quotation}
{ \bf How can we design a privacy preserving scheme to protect user's privacy in OSN?}
\end{quotation}
in the context of today's OSN dependency and the inevitable trend of Graph Search wide deployment.

\section{Research Scope}
After providing the research question of this research, the scope of this research will be defined in this section. Indeed, there are multiple solutions to secure users' privacy in OSN, since the limited time and resource for this research, it will focus on solving privacy issue of Graph Search by addressing the following key technical research points:

 because it can stop the privacy invasion not only from malicious adversaries, but also from untrusted service providers in such an , and try to 

\begin{itemize}
\item{\bf Cryptography in OSN.} Cryptographic solutions for privacy issue are preferred in outsourced cloud service scenario such as OSN. 
\item{\bf Search with Secure Index.} Existing research works had proposed privacy preserving schemes for sensitive data in OSN by using cryptographic solutions \cite{baden2009persona, guha2008noyb, tootoonchian2009lockr, sun2010privacy, backes2011security, feldman2012social, jahid2011easier, nilizadeh2012cachet}. However, they are unsearchable after encryption and not compatible with Graph Search while this research attempts to introduce searchable encryption to satisfy the searchable requirement of privacy preserving scheme over Graph Search.
\item{\bf Extended Search Function for Encrypted Graph.} Previous works proposed several constructions based on searchable encryption to answer the limited queries such as adjacency queries, neighbour queries \cite{chase2010structured} and short distance queries \cite{meng2015grecs} in encrypted graph. To fulfil the requirement of rich search functions in Graph Search, this research will design privacy preserving graph search scheme with boolean queries, spatial queries and ranked queries.
\item {\bf Large Scale Deployment.} Deploying the privacy preserving scheme into a extensive graph to enable private graph search on it is the final objective of this research, a formal security analysis and evaluation in big dataset will be presented to verify its performance in security as well as in efficiency over Big Data. 
\end{itemize}

As a practical privacy preserving scheme is the main targeted artefact of this research, the research will not cover the modelling and algorithm design of searchable encryption scheme itself.















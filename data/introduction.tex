\chapter{Research Problem}
This report outlines the doctoral research on keeping sensitive data privacy in Online Social Network (OSN) by using 'searchable encryption' technique. 
\section{Background of Problem}
OSN is highly popular for many years. Services such as Facebook, Twitter and WeChat allow individuals to create their online profiles, make cyberspace connection with friends via these platforms. 

A result of the huge growth of OSN is more and more sensitive personal data is revealed while OSN users enjoy using it. In 2005, two researchers from Carnegie Mellon University conducted a study in the online behaviour of more than 4,000 students in the university who had signed up in Facebook \cite{gross2005information}. In their study, they found that the majority of Facebook users at CMU provided astonishing amount of sensitive information in their Facebook profiles: it may contain their real names (over $90\%$), date of birth ($87\%$), personal images ($80\%$) and so on; actually, privacy preference settings are provided for OSN user to control the searchability, visibility and access privilege of their profiles, but it is sparingly used, which make them searchable and identifiable to the vast majority of anybody else in the network. 

Due to the richness and variety of sensitive information disclosed in OSN, the privacy of these data is of critical importance to OSN users, as the permissive and unconcerned uses of sensitive information will put themselves at risk for many attacks from the cyberspace or even the real world: the precise personal information helps potential adversaries to construct more deceitful fraud messages (e.g. Context-Aware Spam \cite{brown2008social}); it also can be used to re-identify some anonymous datasets like hospital medical information by matching the common attributes \cite{sweeney2002k}; additionally, OSN users may be easily stalked as they revealed their location or timetable when they shared their activities to their friends.

Although the main purposes of OSN are to help its members to communicate with others and maintain social relations in a more convenient and affordable way, OSN service providers also design diverse Social Network Services(SNS) for its user interaction virtually: it includes updating activities and location information, sharing multimedia (e.g. photo, video) and events, getting updates and comments on activities by friends, etc.. All of these services retain a huge amount of data from its users: until 2013, there were 1.11 billion monthly active users and they put 4.75 billion shared items in Facebook, these contents received over 4.5 billion 'Likes' from their friends \cite{facebook2013growth}. 

Making these data searchable to further enhance to capabilities of OSN had become an interesting topic in recent years. In 2013, Facebook introduced their social network search engine -- Graph Search \cite{facebook2013graph}. This search engine aims to return the information based on the content from user's social network of friends and connections: for example, you may search ``The restaurants visited by friends live in Melbourne'' to get more personally relevant results. The concept of Graph Search is also introduced by other OSN service provider: LinkedIn has Job Search Engine which is based on user's location, skill as well as the information from their colleague and alumni; WeChat also can search user-specific content from Moments and Official Accounts. 

In a nutshell, Graph Search-like search engines are powerful tools as it makes search result . Therefore, by extracting search results from friends, people with similar or the same attributes (e.g. geographical location, hobbies, etc.) are likely to become friend \cite{mcpherson2001birds}. 
\section{Research Questions}

\section{Significance of Research}